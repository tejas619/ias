\documentclass[11pt]{article}
\usepackage{graphicx}
\usepackage{epsfig}
\usepackage{url}
\usepackage[english]{babel}
\usepackage{vmargin}
\usepackage{times}
\usepackage{amssymb}
\usepackage[fleqn]{amsmath}
\usepackage{cite}
\usepackage{titling}
\usepackage{color}
\usepackage{xspace}
\usepackage{listings}
\usepackage{upquote}
\usepackage[hidelinks]{hyperref}
\usepackage{wrapfig}
\setpapersize{USletter}
\usepackage{textpos}
\usepackage{enumitem}

\setlength{\textheight}{220mm}
\setlength{\textwidth}{160mm}
\evensidemargin=1.1in
\oddsidemargin=1.1in
\topmargin=0.45in
\usepackage[compact,noindentafter]{titlesec}
\titlespacing{\section}{0pt}{*2.4}{*1.8}
\titlespacing{\subsection}{0pt}{*2.0}{*1.6}
\titlespacing{\subsubsection}{0pt}{*1.6}{*1.2}

\newcommand{\fulltitle}{Malware Analysis for Proactive Detection and Prevention\xspace}
\title{\fulltitle}
\author{Tejas Khairnar (1207690220, Group Leader) \\Sujay Vaishampayan (1209248166, Deputy Group Leader)\\  Zhibo Sun(1207644187)\\Harshil Maskai (1209174092)\\ Aloma Lopes (1209273126)\\ Chaitanya Palaka (1209261868)\\ Varun Chandrasekar (1209248010) \\ Kunal Bansal (1211213169) \\ Raj Dalvi (1209232176) \\ Vimal Vadivelu (1209203043)\\ \\
		Arizona State University}

\begin{document}
	\begingroup
		\fontsize{15pt}{15pt}\selectfont
		\begin{center}
			CSE 543 Information Assurance and Security \\~\\
			Interim Report
		\end{center}
	\endgroup
	%{\let\newpage\relax\maketitle}
	\tableofcontents
	% This puts the word "Page" right justified above everything else.
	\addtocontents{toc}{~\hfill Page\par}
	
	\section{Introduction}
	\subsection{Background and Motivation}
	In today’s world connected by the Internet, cyber security has become a major concern. In order
to be secure, software as well as hardware industries around the world are working proactively to
secure their software and stay secured. One of the major threats to these industries arises due to
malware which can be defined as a small piece of software that is intended to damage or disable
computers and computer systems. Hence our research survey is focused on studying techniques
to proactively detect and defend against these malwares.

Stuxnet~\cite{creators2013kill, stuxnet} considered to be the most sophisticated piece of malware ever detected was used in operations against Iran in 2010. Stuxnet was so sophisticated and complicated that it was believed that this malware could be developed and deployed only with fundings which rivals that of the military of a nation. Stuxnet is unique in its own nature because unlike other Viruses or Worms it targets systems that are traditionally not connected to the internal network, which can also be termed as isolated machines. It infected Windows machines through USB keys and then propagated across the network scanning for Siemens Step7 software on computers controlling Programmable Logic Controllers (PLC). This helped the malware to use the information it gathered to take control of other systems and crash them upon infection. Stuxnet even provided fake feedback to the main controllers outside the network, leading the other systems to believe that the infected computer is functioning normally. All these efforts were part of a Cyber War waged against the Iranian Nuclear Program in order to slow it down.

The above example inspires us with the need to be aware about cyber security and how it might affect our cyber world. Many organizations like Kaspersky labs, CrowdStrike are constantly developing expertise to detect these kind of malwares and Trojans spreading around the world. Therefore, we found this area,'Malware Analysis for Proactive Detection and Prevention', very intriguing and considered it to be the center of our research survey.
	\subsection{Goals and scope}
	Our goal for this research survey is to explore and scrutinize various techniques to perform Malware Analysis and understand its application in the real world for proactive defense.
	The scope of our research survey encompasses the following areas:
	\begin{itemize}[noitemsep]
		\item{Various methods to perform Malware Analysis}
		\item{Tools currently used to perform Malware Analysis} 
		\item{Techniques for proactive defense against Malware}
		\item{Techniques to prevent IT networks from Malwares}
	\end{itemize}
	
		\section{What is a Malware?}
	Software that deliberately fulfills the harmful intent of an attacker is commonly referred to as malicious software or malware~\cite{moser2007exploring}. These malwares can be further classified into categories such as a 'virus', 'worm' or 'trojan'. Initially, when malware wasn't as rampant as it is today, it was known by just one name, that is a virus. This virus was created by people who wanted to publicly demonstrate their technical ability and skills. As of today, the underlying motivations to create such malicious software has changed. It is no more about displaying an individuals technical skills or having fun with some friends or colleagues. Now, an underground economy has been setup based on these malicious softwares.\\ \\
	Consider a simple scenario which illustrates the distribution of malware and its effects. A bot is a remotely controlled piece of malware that infects a connected network. Now this bot allows any, so called bot-master, to control it remotely. This network of machines which is connected and can be controlled remotely is called a botnet. Now this botnet can be rented or sold by the bot-master to any buyer/user/group to perform malicious activities like spamming email servers, sending spam emails which contain malicious links, content, etc. These links and webpages in turn collect personal information of the victims such as credit card details and bank account credentials. All the people involved in this racket make money by exploiting their victims. The systems are so well connected that it is quite possible to create more sophisticated types of malware which would remain undetected for days, weeks.\\ \\
	The risk described above motivates the need to create a tool that would detect and mitigate attacks by such malwares. In order to counter this risk, softwares like an Anti-Virus which is based on signature-based detection approach have been developed. These signatures are unique to certain malware. This signature based detection approach has two drawbacks. Firstly, these signatures are written by human analysts and are hence prone to human errors. And second, these signatures will only detect known malwares and always fail to detect unkown malwares.
	There are other techniques developed which overcomes the second drawback but these techniques generate a lot of false positives. This means that a lot of times legitimate samples are falsely classified by the detection system as being malicious due to detector's inability to distinguish between malicious software and legitimate software.
	\section{Types of Malwares}
	This section will give you a brief overview about different types of malware programs observed in the world today. 
	\begin{itemize}
	\item\textbf{Worm:} 
	The main characteristic of this kind of malware is reproduction. It continuously multiplies itself once it gets into the network. The first worm which was introduced to the world was called Morris Worm~\cite{spafford1989internet}. Then was Nimda~\cite{machie2001nimda} worm which infected variety of Microsoft machines via an email exploit. 
	\item\textbf{Virus:}
	This is a piece of code that attached itself to other programs, including operating system. These are the programs which requires the host to activate or execute it to spread or perform malicious activities. If this piece of code attaches itself to a shared file on a network then it might infect the whole network.
	\item\textbf{Trojan horse:}
	The name Trojan horse, for this type of malware, comes from the famous historical incident from the city of Troy where the Greeks constructed a huge hollow wooden horse to gain entrance into the city of Troy~\cite{trojanhorse}. This is a type of software which pretends to be useful but performs malicious actions in the background. Once installed, this software might download additional malware or infect other files on the system.
	\item\textbf{Spyware:}
	Software which sends information out of the victims computer is known as Spyware. This information can be anything which interests the attackers like bank account credentials, browsing history and even the webcam snapshots.
	\item\textbf{Bot:}
	A piece of software that allows a Master to control it remotely. The network of connected bots is called a botnet.
	\item\textbf{Rootkit:}
	The main characteristics of a Rootkit is to hide itself from the user of a computer system. This techniques are highly malicious and can be applied at different levels. For example, execution of a malware on a system can be hidden using a rootkit, which would not allow any Anti-virus software to detect the malware even though it has signature detection enabled.
	\end{itemize}
	
	\section{Various Tools and Techniques to perform Malware Analysis}
	In our research paper we surveyed various malware analysis techniques. The following are techniques to perform Malware Analysis.
	\subsection{Using Hardware Virtualization}
	Installing an antivirus on your machine or making sure that there is an active Intrusion Detection or Prevention System is very common nowadays. However, with the evolution of techniques to detect and stop malwares, the malwares themselves have adapted and learned how to prevent themselves from being detected and they enter the system stealthily and unnoticed. The malware will then execute and delete itself from the system without leaving a trace or copy itself on to another system on the same network and continue spreading and affecting other machines. Malware coders usually add a built-in defense to their malware sample in the event an antivirus or IDS is detected in the system. Moreover, the malware could also modify the system or the antivirus itself, making the machine completely vulnerable to any threat. This is where the concept of performing malware analysis using hardware virtualization techniques comes into play. In these techniques, the IDS or antivirus resides in another part of the system which is not visible to the machine or the malware. Thus, it cannot directly modify the IDS code to hide itself or in any way prevent itself from being detected. This field has sparked many new technologies and studies using different Virtual Machine Monitors (VMM) and making use of a technique called Virtual Machine Introspection (VMI) to remotely examine the machine under scrutiny. In this section, we will be looking at some different tools and techniques proposed by various researchers in this field and how they are applicable to the issue at hand.
	\subsubsection{Ether}
	We start of by talking about one of the first tools developed to perform malware analysis with the help of hardware virtualization extension. Ether~\cite{dinaburg2008ether} is an application which utilizes the Intel VT architecture and resides completely outside the machine to be inspected or analyzed. The main goal of this application was to remain transparent and external to the target machine and perform analysis without revealing itself. The paper addresses the following requirements for hiding changes made by Ether to achieve transparency:
	\begin{itemize}
		\item Higher Privilege – The malware analyzer must have a higher privilege than the maximum privilege level that can be obtained by the malware. 
		\item No non-privileged side effects – This means that if the malware analyzer creates some exceptions which induces side effects, then they should occur at a higher privilege level than what is accessible to the malware. This ensures that the analyzer can make necessary changes to the system memory, registers, etc. in order to ensure transparency.
		\item Identical Basic Instruction Execution Semantics
		\item Transparent Exception Handling
		\item Identical Measurement of Time – This means that the timing information recorded by the malware analyzer should be same as that of the normal system. If it is different, then the malware will be able to detect the presence of an analyzer. This is important when an exception handler is triggered by the malware analyzer and changes need to be made to CPU registers in order to maintain transparency.
	\end{itemize}
	Ether utilizes the Xen hypervisor as its base implementation. It utilizes the hardware extension provided by the Intel VT platform to monitor instruction execution, memory writes and system call execution. Ether can also limit the scope of analysis to a single process when needed. In order to maintain transparency in the system which is the most important aspect of a malware analyzer, Ether performs the following tasks:
	\begin{itemize}
		\item Hides the trap flag so that the malware cannot detect its presence
		\item It performs page table modifications to shadow page tables. Thus, the malware is not aware of the presence of shadow page tables and it can closely monitor faults which were created by Ether and by the system itself.
		\item If Ether makes changes to the SYSENTER\_EIP\_MSR register, it stores the previous value and outputs the previous value when asked by the system
	\end{itemize}
	The final implementation of Ether is resilient to many types of attacks which are possible when the malware analysis application is in the guest itself such as an antivirus or IDS. Ether is present in dom0 of the Xen hypervisor and hence if the malware performs memory analysis, it cannot detect Ether. Since Ether returns the original values of the registers even if it makes changes and modifies the values, this type of detection by the malware will fail. Ether also is invulnerable to memory based and privileged instruction handling detection techniques. In order to test the accuracy of Ether, the authors developed two tools based on the original implementation i.e. EtherTrace and EtherUnpack. EtherUnpack performs fine grained tracing and EtherTrace performs coarse grained tracing. EtherUnpack is used as an unpacker for malwares which use obfuscation techniques to hide their presence. EtherUnpack performs single stepping which involves pausing at every instruction and then continuing with the execution. It successfully managed to unpack majority of the samples that were used for testing by the authors. However, there is a significant performance hit while utilizing single stepping and hence Ether is not very effective for real time analysis. Ether trace monitors native system calls in Windows. Ether Trace was able to monitor all traces of the system calls being performed by the malware sample from a synthetic set generated by them. 
	
	In this paper, Ether, a transparent and external malware analysis tool was introduced which leveraged Intel VT and the Xen Hypervisor to perform malware analysis using hardware virtualization extensions. It is resistant to most in memory detection techniques, register based detection techniques, etc. however, it is vulnerable to certain timing attacks which involve external clocks. This was a very good tool initially but was not effective in real time analysis and hence numerous other tools were proposed such as DRAKVUF, SPIDER, V2E and Process Implanting. These tools will be introduced and explained in the upcoming sections.
	
	\subsubsection{Process Implanting}
	The most important aspect of performing malware analysis using hardware virtualization techniques is to perform Virtual Machine Introspection (VMI) to bridge the semantic gap between the host and the virtual system. The ‘Semantic Gap’ problem is between the guest operating system and the host system in a virtual environment. This is basically the interpretation of the state of a guest machine or operating system by only analyzing the bytes of memory in use by the virtual machine. This paper introduces a new concept called Process Implanting~\cite{jiang2011procimplant} which gives us a view of the guest operating system from inside the guest itself, thus somewhat bridging the semantic gap. The general idea behind this concept is to perform VMI by replacing the execution of a running process on the guest operating system by the malware analysis process which will perform virtual machine introspection from inside the virtual machine itself and thus give a direct overview of the entire system. The analysis tasks are hidden from the system view and the machine perceives the process as a normal one doing the tasks assigned to it.
	The process implanting technique deals with four principles of security that are necessary in order to go undetected by the malware which are stealthiness, isolation, robustness and completeness. In traditional systems, the malware detects the presence of an antivirus or intrusion detection system (IDS) and appropriately takes step to ensure that it goes undetected by the antivirus by hiding itself from the process list or posing as another process. Process implanting performs a similar step to achieve stealth. The procedure involves selecting a random process currently being executed in the guest’s system and stealing some time quanta in order to install itself in the system without any other process knowing. The system administrator can even select a process to be implanted at runtime in order to maintain stealth. Multi-processor support if disabled on the guest operating systems so that processes running on other CPU’s cannot detect the presence of the implanted process. The implanted process utilizes the same code and data stack utilized by the victim process so that other processes or the guest operating system itself cannot recognize any changes being made in the system. One of the most important features of the process implanting technique is that it doesn’t use the memory that has been registered by the system for it’s activities. This means that no process in the guest operating system can access the data structures created by this implanted process as the extra memory is not registered by the guest operating system at bootup. This feature satisfies the isolation principles in process implanting Another handy feature is that all libraries are included in the process executable. This removes dependency on the guest operating system for libraries thus maintaining the stealth principle. The implanted process is assigned root privileges and the ‘unkillable’ flag is set in the system to ensure that the malware or any other process cannot directly shut down the process. There is a volley of communication that happens between the implanted process and KVM or hypervisor while shutting down the implanted process. If the execution of the victim process is not resumed correctly, then the malware or guest operating system will realize that there is a fault in the system. It involves the use of a covert channel to set bits which will guide the implanted process on whether to wait for child processes to shut down before exiting. This fulfills the completeness principle as the implanted process leaves no trace in the system and resumes normal execution as if nothing had changed in the guest operating system. 
	The Process Implanting technique was implemented by storing the implanted process image in ELF format. The hypervisor first scans the page tables of the victim process to ensure that the guest kernel has not been subverted and then only moves forward with the implantation process. They successfully tested implementing ‘ltrace’ to trace malware and infected applications in the guest operating system.
	This technique had better performance than ‘Ether’ which was described in the previous section. Moreover, it performed real time analysis and reported live results back to the system. It is a marginal improvement over ‘Ether’ and doesn’t need to stall the system to retrieve information about the system. It almost completely bridges the semantic gap that arises during virtual machine introspection. One drawback that is evident from the implementation details of this technique is what would happen if another process in the guest operating system passes the control to this process and expects some output in order to proceed. Would the hypervisor restore the original working process and remove the implanted process or is there any mechanism at all to detect this condition? This situation would easily leave the entire system in a deadlock or would lead to timeout of other processes and eventually the malware would understand that there is something weird about the victim process and realize that it is a malware analyzer. Next, we will talk about V2E, a tool that combines hardware virtualization extensions and software emulation in order to perform transparent and extensible malware analysis. 
	
	\subsubsection{V2E}
	We now shift our focus on a tool which achieves the same transparency principle on which ‘Ether’ was founded but along with the added support for code instrumentation. ‘V2E’~\cite{yan2012v2e} is a protocol developed by Yan, Jayachandra, Zhang and Yin which combines hardware virtualization along with software emulation to create a transparent and extensive malware analysis platform. In this protocol, the malware analyst executes the malware sample on the guest operating system which runs using hardware virtualization and records all the modifications that take place in the system during its execution. It then replays these modifications in a software emulator in the exact same way as it happened in the guest operating system except this time with the support of code instrumentation techniques which allows for fine grained analysis to take place also referred to as dynamic binary analysis.\\ \\
	The protocol divides its operation between two realms i.e. the recording realm and the replaying realm. These two realms combined form the entire guest operating system. The recorder records all the events and states of the guest operating system and forwards it to the replayer in order to get the exact same execution that took place when the malware was active in the system. One major task in this approach was to decide which and how many events and states should be recorded. If too many states and events are recorded, then the performance will be same as ‘Ether’ which involves single stepping. On the other hand, if too few events and states are recorded, then the exact execution sequences and changes made by the malware to the operating system may not be able to be reproduced accurately in the replayer. In order to overcome the previous hurdle, every operations and instructions are classified into multiple categories and each one is handled in a different way to ensure accurate replay. The authors built the recording realm on the KVM Hypervisor and the replayer realm using TEMU which supports dynamic binary malware analysis. They modified a few functions of TEMU in order to achieve transparency and increase the analysis efficiency. The four principles of the architecture of V2E are transparency, instrumentation support, efficiency and an adjustable view. We have covered the first three in previous sections. The fourth one means that the architecture does not focus on the execution of the system as a whole, but it focusses on the malware’s behavior primarily.\\ \\
	The malware resides in the recording realm and the guest operating system and other general applications reside in the main realm. Since hardware virtualization works on the principle of Two Dimensional Paging (TDP), the system calls and execution is trapped by leveraging the page faults that occur and by triggering VMExit events which are transparent to the guest operating system. The application utilizes virtual machine introspection to bridge the semantic gap. The kernel data is parsed to extract the relative physical addresses of the OS data structures and semantics like the current process list, file objects, etc. and then the addresses are mapped to host physical addresses with the help of TDP. In order to avoid utilizing an external clock during recording, the authors have proposed to use a shadow time stamp counter which calculates an estimate of how much time the guest is actually running. The recording model involves the KVM hypervisor and writing logs to the file system. They implemented a custom version of the ‘mmu\_notifier’ script and modified the ‘mmu.c’ file to add realm control and enforcement logic to it. The replayer made use of the dynamic binary translation platform, Bitblaze, and implemented the software Translation Lookaside Buffer for faster memory management to improve efficiency. The results of their evaluation and simulation showed that V2E accurately hid from multiple emulation detection techniques that malwares employ to ensure their existence remains hidden. Moreover, on testing with real world malware samples, V2E was able to record and replay the exact behavior exhibited by those malwares. 
	Looking at the results published by the authors, the performance was definitely better than ‘Ether’ but not good enough. It took ‘V2E’ 52s to decompress the linux source code when recording was enabled and 3 seconds without recording. Similarly, it took V2E 2.5 seconds for a page in Internet Explorer to load when the recorder was not running and 13.8 seconds otherwise. On comparing this with the single stepping technique, it had a performance improvement of over 100 times. This was a welcome change and involved performing external malware analysis as the guest operating system was in another realm. The protocol assumes that all common instructions are executed correctly, however, if the malware exploits this bug, then V2E will not be able to successfully replay the malware execution.
	
	
	\subsubsection{DRAKVUF}
	We now shift our focus on to DRAKVUF~\cite{lengyel2014scalability}, a modern malware analysis tool, which proposes a novel approach in which we make use of hardware virtualization extension such as the Xen virtual Machine Monitor (VMM). We run multiple virtual machines on this hypervisor and the memory of each virtual machine is analyzed in the 'dom0' of the Xen hypervisor. DRAKVUF makes use of Copy-on-Write method to examine the memory by efficiently transferring the part of the memory which it wishes to examine over the network. The virtual machines are all connected along the same network via a NAT engine which makes the virtual machines invisible to each other. This effectively reduces the number of attack points or intrusion points for the malware and each virtual machine functions as an independent node. DRAKVUF makes use of a Direct Memory Access technique using LibVMI to access the memory of the target machine.\\ \\
	The paper achieves dynamic malware analysis by enabling stealth, fidelity, scalability and isolation of the monitoring technology from the affected system. DRAKVUF implements the use of break point injection in the operating system in order to log events and the values of various registers and the memory addresses being referenced. Inserting a break point in the system by writing a INT3 command in areas of interest triggers a VMEXIT command which transfers control to 'dom0' which is where DRAKVUF is running. The Extended Page Tables are then copied and examined for memory references that shouldn't be there, etc. DRAKVUF enables execution tracing which is the ability to trace the execution of processes by monitoring system calls. Initially, DRAKVUF makes use of Rekall to access the Kernel Processor Control Region (KPCR) which gives us direct access to various kernel symbols used in the operating system. Using the addresses obtained from the Extended Page Table and the offsets from the KPCR, we can directly get the memory location of the EPROCESS block that we need to examine. DRAKVUF also tackles Direct Kernel Object Manipulation (DKOM) attacks by directly tracking kernel heap allocations using the breakpoint injections. Thus, we can find out the addresses at which Windows is allocating memory for structures and extract the return address of the calling thread from the heap and eventually find the location of the block which is unhooked from the system view. DRAKVUF also allows us to monitor file system accesses using memory events. The tool traps the instructions being executed to create a \_FILE\_OBJECT during the creation of a file in memory and extracts the allocation address from the heap. This in turn gives us direct access to the \_FILE\_OBJECT of the corresponding file and we can easily access the data. DRAKVUF allows extracting deleted files from memory as long as those memory blocks are not yet overwritten by the operating system. By trapping file deletion kernel calls by the operating system, the we can retrieve the address of the handle to the \_FILE\_OBJECT. On examining the handle table of the process, we can eventually find the \_FILE\_OBJECT block and retrieve it from the memory using Volatility tool.\\ \\
	Implementing a dynamic malware analysis system using hardware virtualization extensions along with virtual machine introspection is a novel approach. This enables almost complete isolation of the malware and the IDS which greatly increases the probability of identifying threats.	
	
	
	\subsubsection{SPIDER}
	Let us move on to the last tool that we will be discussing in this category of malware analysis. Deng, Zhang and Xu, from Purdue University, developed a malware analysis tool, SPIDER~\cite{dongyan2013spider}, short for Stealthy Binary Program Instrumentation and Debugging via Hardware Virtualization. The main aim behind this tool was to be able to trap the execution of an application at particular targeted instructions without tipping off the malware about it’s presence. The authors have developed a new mechanism called as an invisible breakpoint which follows the principles of efficiency and flexibility and leverages the Virtual Machine Monitor to hide itself from the guest operating system. It achieves this by splitting the ode and data view as seen by the guest operating system along with data watchpoints.\\ \\
	In traditional software breakpoints, a 1-byte instruction (INT3) is executed which triggers a Breakpoint exception, thus switching the flow of control to the hypervisor where the malware analysis tool lies. The drawback in this situation is that the malware can easily detect that the instructions have been modified to switch execution control to the hypervisor and it will hide its presence. The other method is to utilize hardware breakpoints to trigger breakpoints, however, it is possible to use up all hardware breakpoints via software programming, thus leaving the debugger with no available hardware debuggers to use. SPIDER follows four principles to remain undetected in the presence of malware. They are flexibility, efficiency, transparency and reliability. In order to maintain transparency, SPIDER utilizes the Extended Page Table (EPT) to split the code and data views. If a breakpoint needs to be triggered, then only the code view of the guest is modified and the data view is left as it is. Thus, the guest cannot find a difference in the instruction as the code view is inherently modified. Once the breakpoint is triggered, the exception handler handles the exception transparently such that the flow of control in the guest would be same as that without a breakpoint. Also, it modifies the code and data view as it would be without the breakpoint in order to ensure transparency and makes sure that the guest cannot overwrite the breakpoint by making use of higher privileges than the rest of the guest system. SPIDER splits the code and data view by mapping them to two different host physical pages with ‘mutually exclusive attributes’. The code view is executable but cannot be read and the data view is readable but cannot be executed. Writing is disabled on the both the code and data views to enable easy breakpoint handling and to ensure that no malicious program interferes with the execution of the program. As in order implementations of tools that utilize hardware virtualization extensions, the breakpoint handling is always done by the Virtual Machine Monitor in order to ensure transparency and so that the malware cannot detect presence of an external entity that is monitoring the system. \\ \\
	Since writing to the code and data view is disabled, when the malware in the guest operating system tries to write to the data view, it will receive an error and will not be able to write. The malware can then easily figure out that it is being tracked and will hide its presence. Thus, SPIDER synchronizes changes from the data view with the code view so that the malware cannot detect the presence of an analyzer. SPIDER captures the EPT violations and records the length of the data to be written and the address it is to be written at. It then makes the appropriate modifications to the code view. Any breakpoints that were added to the code view need to be updated if they were overwritten or if they are not meaningful any more. Since handling of breakpoints triggers the flow of control to bounce back and forth between the hypervisor and guest system, there are timing issues that may arise which may be leveraged by the malware to detect the presence of an analyzer. The solution implemented by the authors involves tracking the Time Stamp Counter (TSC) and hides the CPU cycles spent in transferring the control back and forth and thus, the guest believes that it did not lose any time. The authors have implemented SPIDER on the KVM hypervisor.
	The SPIDER malware analysis tool was pit against OllyDbg and IDA Pro in order to monitor its performance and if the malware can detect its presence or not. After executing a process ‘hostname.exe’, which stores the string of the name of the host and also has anti-instrumentation and anti-debugging code built in to it, OllyDbg and IDA Pro failed to Trap the string successfully, where as SPIDER was able to as it ran undetected. On running performance evaluations, the original KVM implementation ran a 98.7MB file in 4171 ms while the KVM implementation with SPIDER ran the same file in 4192 ms. This shows that this is one of the fastest malware analysis tools which utilizes hardware virtualization extensions as of 2013. It provides multiple advantages over ‘Ether’ and ‘V2E’ as mentioned earlier and has efficiency similar to that of DRAKVUF. This may well be one of the best external malware analysis tools so far and in the future many more will be developed. \\ \\
	In this section, we have seen multiple tools that perform malware analysis via hardware virtualization extensions. We have seen how the first successful tool ‘Ether’ performed and the modifications that were made to create more efficient and robust tools like ‘DRAKVUF’ and ‘SPIDER’. This area of malware analysis is still under research and with the development of the cloud environment and continuous enhancement of the current virtualization systems, many more advanced tools will be released in the future. Compared to traditional methods to monitor malware, hardware virtualization might be the way to go currently as a lot of computation is being shifted to the cloud. The drawback of these techniques is that you need to have access to the entire memory of the guest system from the host which means that the memory needs to be on the same system as the host. However, in the future implementations will be released where the system where the host resides will be different than the location of the guest memory and it could be accessed via a Software Defined Network.
	
	\subsection{Static Malware Analysis}
	\subsection{Dynamic Malware Analysis}
	\subsection{Malware Reverse Engineering}
	
	\section{Techniques to prevent IT Networks from Malwares}
	\subsection{Distributed Intrusion Detection System}
	With the development and accessibility to new and cheaper technology, there is a proliferation of heterogeneous computer networks which may add difficulty to intrusion detection system. With greater connectivity, the outsiders have more opportunity for accessing vulnerable systems and insiders can avoid detection. Secondly, Network user identification problem where a user can move across systems to defy the current Intrusion detection system's adds another level of attack vector. Problem with using a single point of audit is that multiple host computers can produce large amount of data which needs to be analyzed by a single IDS (Intrusion Detection System) machine and creates an overhead. The attacks that a normal IDS can’t detect includes, an attacker trying to discover insufficiently protected system where the attack activity is lower than the once that can be detected by IDS, i.e. Using the technique of diffusion. Secondly, an attacker can try to access the files using multiple compromised hosts, which can have detected when data from multiple sources is aggregated and correlated. There is always a trade-off between sending limited data from host to IDS versus an attack that could be missed.
	The Distributed Intrusion Detection System architecture combines distributed monitoring, data reduction and centralized data analysis. A Distributed Intrusion Detection System has following features:
	\begin{itemize}
		\item The host and LAN monitors are used to collect evidence of unauthorized or suspicious activities and DIDS director is responsible for analysis of collaborated data
		\item The architecture is responsible for monitoring bidirectional communications between DIDS director and host machines, which includes notable events and anomaly reports. The director can access more detailed report based on an event and can also update monitoring capabilities on individual hosts.
		\item A large amount of packet filtering is performed at host level to minimize additional network bandwidth.
	\end{itemize}
	There are various components involved in a Distributed Intrusion Detection System. The \textit{DIDS Director}, consists of three components that are logically independent processes and can be hosted in distributed or centralized system. The communication manager, is responsible for moving data between each host, director and LAN monitors. It receives data or make requests with host and LAN monitors. Expert system, receives the data collected by the communication manager from hosts and LAN monitors. It evaluates and reports the security state of individual hosts and the overall network system. User Interface, provides a System Security Officer an interface to watch activities of hosts and network as whole. It can also be used to request specific information about events, host or LAN. The Network-user IDentification (NID), provides a solution to multiple user identity problem, where an attacker can move across hosts to prevent detection in the system. It creates a unique Network User Identification for every new user that enters the system and same NID is used to track activities of the user in the monitored environment. The Host Monitor, watches audit records of transactions on the system that includes file access, system calls, process execution and logins. The host monitor decides whether the transaction needs to be forwarded to the expert system for further evaluation. Critical records are forwarded automatically; others are processed by the local monitors on the hosts. It creates an abstract object called an event to define an activity. Each event is associated with a corresponding action and a domain. A subset of events is forwarded to Expert system for further analysis. The LAN Monitor, observes each packet on its part of the LAN and uses them to construct higher level objects such as connections and service requests using the TCP/IP Protocol or UDP/IP Protocols. It audits host to host connections, network services and traffic volume per connection. It creates profiles of expected network behavior. It also uses data analysis and correlation heuristics to identify possible intrusive behavior of any individual connection. It also helps in creation and use of Network Identifications (NIDs). The Expert System, uses a rule based system where rules are derived from hierarchical \textit{Intrusion Detection Model}. It describes transformation from raw data to higher level abstraction used in describing an attack on a network of computers. The Intrusion Detection Model Process is further divided into six layers:
	\begin{enumerate}
		\item The audit records are received from the host monitors of each individual host operating system and LAN monitors.
		\item The events are created for the audit data which are syntactically and semantically independent of source standard format.
		\item IDM identifies subject from data. A subject defines single identification for a user across several hosts on the network. The subjects are assigned respective Network Identifications (NIDs). After this layer, the entities are defined using subjects and local identification is lost.
		\item At this layer, the subject and event are identified with a context. The set of events are identified as temporal context or spatial context.
		\item This layer evaluates threats to the network and the hosts connected to it. Events are aggregated and combined to define abuses. Abuses are divided into attacks, misuses and suspicious acts. The targets of abuse are identified as system objects or user objects and as being active or passive.
		\item The model produces a numeric value from 1 to 100 which represents overall security state of the network system.
	\end{enumerate}
	Using the Distributed Intrusion Detection System, we can leverage LAN structure to monitor user behavior for attacks against the system and can identify attackers moving across multiple host machines using segregation and aggregation of data from multiple host machines. Also, filtering data at hosts provide optimized learning of the Expert System. This system can be extended to large networks.
	\subsection{SNORT}
	Most of the current Network Intrusion Detection systems (NIDS) for malware analysis face the basic problem that the updates are released at regular intervals of time and malwares are detected or analysed at uneven intervals of time as a result of which systems are vulnerable for phases or intervals. The customers have to wait for vendors to release updates to the vulnerabilities. Secondly, Commercial Network Intrusion Detection Systems are very expensive which makes them inefficient for smaller business networks and devices. With the imminent growth of Internet of Things (IoT) devices there is a need of lighweight and cross platform resources.
	Snort is a lightweight and cross platform NIDS that can be used for both monitoring network packet data and create an inference based on the traffic to define an attack. The administrator can take define suitable actions, manually or automatically, based upon the alerts.
	A Snort system consists of following components:
	\begin{itemize}
		\item The Packet Decoder
		\item The Detection Engine
		\item The Logging/ Alerting Subsystem
	\end{itemize}
	This further serves many features. Snort is a lightweight and cross platform system allows deployment on any of the nodes, from a large system like mainframe to a small IoT device like raspberry Pi. The small system footprint and easy configurability allows easy implementation of specific security solutions in short time. Rule based logging allows pattern recognition using advanced algorithms allows a variety of attack detection like buffer overflow, SMB probes, CGI attack with creating proper alerts in the respective domain like Server Message Block, Win Popup or an alert file. It decodes the packet from data link layer to the application layer, which allows rules could be implemented on data. This helps in detecting hostile activity including CGI scans, buffer overflows, or unique malware payload using finger print matching.
	Thus considering all the features snort has to offer, we can use it as a Network Intrusion Detection System which is small and flexible and can be used
	
	\subsection{Improving Network Management with Software Defined Networking}
	We can use Software Defined Networking (SDN)~\cite{kim2013improving} to counter the problems of network management and configuration. This is a proactive approach of prevention of malware using vulnerabilities in the network. It works by separating the data plane and the the control plane making switches as forwarders and using a logically centralized program to control the behavior and policies of the entire network. In Software Defined Networks, a central program (controller), defines the behaviour of the whole network (control plane) and network devices become simple packet forwarding tools (data plane).
	A software defined networking provides following features:
	\begin{itemize}
		\item Centralized management over the distributed management
		\item Centralized controls allows network configuration changes in a controlled way
	\end{itemize}
	 A software defined network architecture can be implemented using Procera[3], which offers event driven network control framework. A Procera system is developed over a functional reactive programming where policies can be expressed in high level language.We can create a network environment that prevents attack by using data from multiple control domains like time, data usage, privileges for user in the network and flow or network behaviors where network is considered a single identity. 
	 A Software Defined Network like Procera has following components:
	 \begin{itemize}
	 	\item Event Sources, create dynamic events for controller, Intrusion Detection System (IDS) and authentication Systems. The structure of events is defined by corresponding parser in Policy Engine.
		\item The Policy Engine, parses network policy defined by the Policy language. The behavior of policy engine is defined by the policy language and the asynchronous events created by the event sources.
		\item Network Controller, is the central control that translates network policy to actual forwarding rules and manages them over OpenFlow capable switch using openFlow protocol.
	 \end{itemize}
		Thus using this technique, we can accomplish optimized and secured low level configuration of devices in a network. This can also prevent a malware that has infected a single host to infect the whole network as point of breach could be detected by mismatch of policies at the network Controller. Thus, provides a preventive method for malware to move across the network and infect other systems.
	
	\subsection{Malware Detection using Analysis of deviations in Application Network Behavior}
	The intend of this paper~\cite{shabtai2014mobile} is to identify malicious applications on a mobile device, identify add on that could have infected a genuine application or malwares that have self updating capabilities which cannot be detected by standard signature matching approach. This malware detection technique has following features:
	\begin{itemize}
		\item It uses the patterns in network traffic which are specific to each application.
		\item It can rate the levels of deviation from normal behaviors.
		\item It can detect malware that change their signatures with regular updates, dynamic loading of compiled android code, dynamic loading of shared object file or dynamic loading of a shared file.
		\item It can run on mobile devices without much overhead on performance.
	\end{itemize}
	This technique has a stepwise approach and functions according to these components. Graphical User Interface, provides an interface for communication with the user and allow the application configuration based on user requirement. Alert Manager, to manage alerts with user. Feature Extraction, tracks the in/ out data from the device on each type of network in different application states with a track on application’s activity or modification time. Feature Aggregator, provides concise data based upon data extracted in the previous step. It uses average, standard deviation, min-max of sent or recieved of transmitted data. Local learner, uses semi supervised anomaly detection technique. It employs cross feature analysis to find correlation between data and assumes that a strong correlation means a normal behavior of the application. Anomaly Detection, classifies network behavior based on the correlation data created in the previous step. If there is high correlation in between data of different features, it can be tagged as a normal behavior. If there is low correlation between data, it can be tagged as an anomalous behavior.
	Using this technique, a malware can be detected by monitoring the network traffic in a mobile device. Even if the malware was not detected previously and locked using its specific signatures, we will still be able to detect it using its network behavior. This solution can be used to protect thin clients from malware attacks.
	
	\subsection{Network Intrusion Detection and Countermeasure Selection in Virtual Network Systems}
	This technique provides analysis about using network level activities to find unique vectors that can be used to back-up insights provided by system level activities. These unique behaviors can be further used to collect and classify information and eventually mitigate malicious software~\cite{chung2013nice}.
	They have created an in-depth analysis of malware network behaviour using the Sandnet system over a period of 12 months. Sandnet counters two limitations, short analysis period and lack of detailed network behavior analysis which provides an overview of network protocols used by the malware, namely DNS and HTTP. 
	The NICE framework provides following features:
	\begin{itemize}
		\item A distributed network intrusion detection and prevention method in virtual environment using the network traffic of the cloud.
		\item It provides a solution for contingency in case of attack or malware is detected to have compromised the systems in the cloud. It can improve the attack detection probability by aggregating data of network as a single identity.
		\item It uses the attack graph model, where each node represents a system in the cloud and connections are represented using edges. Threats are detected by correlating data from events or activities over the cloud in an attack workflow and hence, suggest effective counter measures based on that.
		\item It adds lesser overhead as compared to other proxy based solutions.
		\item It considers attack as an insider or an external breach of information. Hence, provides more secure solution.
		\item It uses a VM protection model, to maintain a state diagram of the current states of virtual machines in the cloud. Each machine in the network is assigned a stable, vulnerable, exploited or a zombie state.
	\end{itemize}
	This framework works on the following components. Network Intrusion Detection System agent, is used to sniff traffic across the bridges in a cloud server. VM profiler, creates an attack graph for the infrastructure, populating specific alert in NICE agent and records the traffic pattern in a network controller. Attack Analyzer, provides attack correlation and analysis operation based upon the attack graph and logs the threat information to the network controller based on analysis. Network Controller, implements the counter measure based upon the data from attack analyzer by virtual network reconfiguration using the OpenFlow protocol.
	The technique provides a Sandnet system for data collection using network analysis which can be used for prolonged period of time. A network activities overview of 100,000 malwares and augment the data with previous efforts. And an in-depth analysis of DNS and HTTP traffic to define protocol specific usage behaviours of malware. 
	
	
	\section{Techniques to perform Malware Anlysis for Malwares on Android Devices}
	Google’s Android System is a comprehensive software framework targeted towards smart mobile devices which includes an operating system based on the popular Linux OS, middleware and a set of key applications. It is an open-source and community based framework which provides APIs to most of its software and hardware components. Specifically, it allows third-party developers to develop their own applications. The applications are written in the Java programming language based on the APIs provided by the Android Software Development Kit (SDK) and compiled to Dalvik bytecode, but developers can also develop and modify kernel-based functionalities as well.
	Due to the ubiquity and prevalence of smart mobile phones, of which Android has a tremendous 88\% market share, it has become a frequent target of malicious attacks through malware, spyware, viruses, trojans, etc, under the guise of legitimate applications. This necessitates research in providing frameworks and systems to detect and protect our smartphones.
	Various methods have been devised in this regard. Two commonly seen approaches are static and dynamic analysis of malware. The former examines the malware without running the application, using methods of source code and binary analysis for hints of malicious intent, whereas the latter executes the malware in a controlled and monitored environment to observe its behaviour. In this part of our project, we will survey a few papers which have proposed techniques and systems to achieve the aforementioned and discuss their methods. 
	
	\subsection{Andromaly}
	Andromaly~\cite{shabtai2012andromaly} is a framework for detecting malware on Android mobile devices which uses a Host-based Malware Detection system that continuously monitors various features and events obtained from the device and applies Machine Learning techniques to detect anomalies by classifying the data as either normal or abnormal. Due to the ‘system-centric’ security model of Android, in which applications statically identify the permissions that govern the rights to their data, the application/developer has a limited ability after installation to dictate to whom those rights are given to or how they are used at a later time. To overcome this limitation, the authors have proposed a lightweight malware-detection system to assist users in detecting suspicious activities on their phones. 
	The detection process consists of real-time and continuous monitoring, collection, preprocessing and analysis of various system metrics such as CPU consumption, number of sent packets through WiFi, number of running processes, battery level, and system usage parameters (e.g., keyboard presses, application startup, etc). After the collection and preprocessing stages, the metrics are sent for analysis to detection units termed ‘processors’ each of which employs its own methods for detecting malicious behavior and outputs a ‘threat assessment’. After this, a notification alerting the user is displayed if malicious behavior is found. The GUI portion of the framework then provides the user with automatic or manual actions which can be chosen to mitigate the threat.
	The main components of the system are discussed in more detail below:
	\begin{itemize}
		\item Feature Extractors – These communicate with various components of the Android framework, including the kernel and Application framework layer in order to collect feature metrics.
		\item Feature Manager -  This component triggers the feature extractors and requests new feature measurements every predefined time interval. It may also apply some preprocessing on the raw features which are collected by the extractors. 
		\item Processor – this component is an analysis and detection unit. It should be provided as a pluggable and modular external component which can be seamlessly installed and uninstalled. Its duty is to receive the feature vectors from the Main service, analyze them and output threat assessments to the Threat Weighting Unit. The processors can be rule based, knowledge based, or they can be classifiers/anomaly detectors which utilize Machine Learning methods.
		\item Threat Weighting Unit – this component obtains the results of analysis by the processors and applies a composite algorithm in order to derive a final singular decision regarding the Android device’s infection level. 
		\item Alert Manager – this component receives the final ranking produced by the Threat Weighting Unit and subsequently applies a smoothing function in order to provide a more persistent alert and avoid instantaneous false alarms. Examples of smoothing functions are : moving average and leaky bucket.
		\item Main Service – this component is undoubtedly the most important component. It synchronizes features collection, malware detection, and the alert process and manages the flow of the whole system by requesting new samples of features and sending newly sampled metrics to the processors.
		\item Graphical User Interface (GUI) – this component provides the user with the means to configure the entire system’s parameters, visual alerting, and visual exploration of collected data.
	\end{itemize}
		The proposed framework employs Machine Learning classification methods in the ‘Processors’ component of the system. Under this approach, the malware detector/classifier continuously monitors various features and events obtained from the system and decides whether the observation samples are normal or abnormal. The classifiers used in the paper for experimenting are : k-Means, Logistic Regression, Histograms, Decision Tree, Bayesian Networks, and Naive Bayes. The evaluations of learned classifiers is split into two steps: training and testing, making the method a supervised learning  form of Machine Learning. During the training phase, a set of both benign and malicious features vectors are provided to the system with their correct labels and the classifier is trained on them. During the testing phase, a different collection (testing set) of feature vectors is evaluated by the classifier and  attempts to differentiate between the benign and malicious vectors.
		Another aspect to consider are the problems of over fitting, reduced generality, increased model complexity and run time which occur due to improper selection of the extracted features. This occurs when there are too many features which are redundant or irrelevant. Because of this, proper feature selection algorithms must be used to obtain high levels of accuracy. This paper uses the filter approach to feature selection due to its fast execution time and generalization ability.
		In conclusion, Andromaly is a good approach to tackling the issues of Android malware and the use of Machine Learning algorithms is definitely a viable approach. Feature Extractors were found to be in the critical execution loop and must be optimized aggressively. Furthermore, the detection of anomalies could be even more effective with the use of a smaller number of features and simpler detection algorithms which ensure stringent resource constraints.

	\subsection{AppsPlayground}
	AppsPlayground~\cite{rastogi2013appsplayground} is a framework that automates the analysis of smartphone applications, specifically targeted towards the Android system. It comprises of multiples components each of which employ different detection and automatic exploration techniques. AppsPlayground is meant to analyze applications for both malware and grayware. Grayware refers to applications which are not malicious but can be annoying, for example by leaking private information for a legitimate purpose without the users awareness. AppsPlayground offers a modular solution to these issues by using a multitude of detection techniques: taint-tracing, sensitive API monitoring, and kernel level system call monitoring.	
	We now describe the overall architecture of AppsPlayground. The general framework is built as a virtual machine environment by re-purposing the Android emulator (Qemu) for a dynamic analysis environment. The virtualized environment is essential for providing scalability which is generally not feasible while using real devices. Some of the challenges in designing such a system are: modeling the GUI, devising an efficient exploration strategy (due to the large number of unique program states which an application will generally contain), and context determination.
	AppsPlayground employs various detection techniques:
	\begin{itemize}
		\item Taint Tracing – Playground uses this technique to track privacy sensitive information leakage. It uses a slightly modified version of TaintDroid (an opensource, high performance taint tracing system for Android), which works for Dalvik bytecode only.
		\item Sensitive API monitoring – playground monitors a few system APIs for detecting possibly malicious functionality, for example, the SMS API, which happens to be one of the most exploited APIs for Android. Malicious applications can use this API to send messages to premium rate numbers without the user’s awareness. Playground can record the destination and content of the SMS messages sent by an app, thereby protecting the device by allowing the user to monitor any malicious activity. Similarly, Playground monitors the Java Reflection API and dynamic bytecode loading.
		\item Kernel Level monitoring – this technique is used by Playground to identify known root-exploits. The method it uses is based on vulnerability conditions and is thus immune to code polymorphism. Some examples of root-exploits already found on Android are: Rageagainstthecage/Zimperlich, Exploid, and Gingerbreak.
	\end{itemize}
	The automatic exploration techniques which AppsPlayground uses are: event triggering and intelligent execution. Many API elements in android are event based, which means that applications may register some code to be triggered only when a certain event occurs. Many malicious applications have been found to register for specific events. AppsPlayground will automatically check that this will not happen. AppsPlayground also intelligently drives the user interface of a smartphone application by dynamically defining and exploring models created from window and widget features. It extracts features from displayed user interfaces to iteratively define a model that approximates the application’s logic. It then creates associations between the current features and those extracted earlier using a method known as widget tracking. Search optimizations are also employed to speed up the process by reducing the search space. Playground also uses sequencing policies to determine the next GUI action. Overall, the implementation of AppsPlayground consists of 3000 lines of Java code.
	AppsPlayground was intensively tested in all sorts of areas, including in detecting information leaks and known Android malware such as FakePlayer, DroidDream, and DroidKungFu and was very effective in all of them.
	
	\subsection{DroidScope}
	DroidScope~\cite{yan2012droidscope} is an Android analysis platform which runs on a virtual Android environment. It is a dynamic binary instrumentation tool which can be modified with plugins as the user intends, with the purpose of analyzing various types of malware and other malicious applications. There are two main reasons for choosing virtualization-based analysis. Since the analysis runs beneath the entire virtual machine, it is able to analyze even the most privileged attacks in the kernel. Also, since the analysis is performed externally, its very difficult for an attack within the virtual environment to disrupt the analysis. It is not without a downside, however, and that is the loss of semantic contextual information when the analysis component is moved out of the box. Also, to enable the virtualization based analysis approach, there needs to be a mechanism to construct semantic knowledge at two levels: at the OS level, and the Java level. With these points in mind, the authors of this paper have designed and implemented DroidScope, which is built on top of QEMU (an Android emulator) which is able to reconstruct both aforementioned semantic views from the outside. DroidScope also provides a set of APIs to help analysts implement custom analysis plugins for whatever purpose they wish.
	The entire Android system, including the malware, runs on top of an emulator, and analysis is performed from the outside. To ensure the best compatibility with virtual Android devices, the authors extended the QEMU by reconstructing the OS level and Java level views simultaneously, implementing dynamic taint analysis, and providing an analysis interface so that future users can build their own custom tools. Several other analysis tools were also developed, such as the API tracer, the native instruction tracer, Dalvik instruction tracer and the taint tracker.
	As mentioned, DroidScope provides a set of APIS to facilitate custom analysis tool development. This is part of an event based interface useful for instrumentation. The APIs provide instrumentation on separate levels: native, OS and Dalvik, so as to mirror the context levels of an actual Android device, and each provides the ability to register callbacks for different events and one can also query or set different kinds of controls. To demonstrate the capability of DroidScope and the types of plugins future users can create themselves, the authors have written 4 of their own analysis plugins: an API tracer, a Native instruction tracer, a Dalvik instruction tracer and a Taint tracker.
	
	\subsection{MamaDroid}
	This technique presents a novel malware detection system for Android, named MamaDroid, that relies on the sequence of abstracted API calls performed by an app rather than their use of frequency, and aims to capture the behavioral model of the app. A key issue that is addressed in this approach is providing resilience to API changes (which often happens) by abstracting specific API calls to either its package name of family name.
	After abstracting the calls, MamaDroid analyzes the sequence of API calls performed by an application, attempting to model its behaviour. This is due to the assumption that malware may use the same calls as an ordinary app, but in a different order and for different operations. MamaDroid then  builds a statistical model representing the transitions between the API calls and model these as Markov chains which can then further be used to extract features and perform classification. Another thing to note, is that given the two separate types of abstraction (package or family), both need separate modes of operation. The authors test both extensively and conclude that the abstraction to family is more lightweight, whereas the abstraction to package is more fine-grained.
	The operation of MamaDroid goes through four phases:
	\begin{enumerate}
		\item Extracting the call graph from each app by using static analysis
		\item Obtaining the sequences of API calls using all unique nodes in the call graph and associating each node to all of its children nodes
		\item At this stage, it abstracts the API calls to either package or family and then building on these sequences, it constructs a Markov chain model
		\item Classifies the app as either benign or malware using a machine learning classifier with the transition probabilities used as the feature vector. 
	\end{enumerate}
	For performing classification using Machine Learning, a number of different algorithms were used: Random Forests, 1-Nearest-Neighbor, 3-Nearest-Neighbor, and Support Vector Machines. After extensive testing and experimenting, the results are conclusive that modeling the sequence of API calls as as Markov chain is a successful method of capturing the behavioral model of that application. This fact allowed MamaDroid to obtain a very high accuracy of classification and it promises to retain that level of accuracy over the years, due to the abstraction of specific API calls.
	
	We have seen that there exist many different approaches to malware detection on Android systems. Ubiquitously, dynamic analysis is considered to be far more advantageous than static analysis on this system, particularly due to the ease of code obfuscation. Another trend we can observe is the use of Machine Learning to learn the various models of the types of malware in operation currently. This idea has a lot of potential and I believe with the progress of machine learning techniques and algorithms, malware analysis will continue to become better and better and perhaps malware on Android systems will be vanquished entirely.

	\section{Techniques for Proactive Defense against Malwares}
	
	Traditional behavioral analysis tends to be a step behind the attackers. To defend against cyber attacks, enterprises have set up response teams but they are usually not effective enough. Sometimes attackers use APT (Advanced Persistent Threat) that is, gaining information on the security measures of the attacked area and elsewhere~\cite{cyberthreatanalysis} by repeated target attacks. As a result, researchers focus on handling APT proactively. Malwares use hooks to register their program into the location~\cite{hookscout}. Rootkits, Network sniffers and other harmful programs use hooks to start their invasion. So, logically, the researchers decided to create a binary-centric hook detection policy which would proactively take care of malware itself. An ideal way to protect from malware attacks is using machine learning and soft computing which is done by Raman Singh and his team of researchers~\cite{softcomputingproactive}. Techniques such as Artificial Neural Networks, Artificial Immune System (AIS) and Fuzzy Set among others were used to combat malware and viruses.  Finally, this report also looked at Muneesh Sharma and Tajinder Kaur’s research on Network Intrusion Detection using Proactive Mechanism~\cite{nidsproactive}. Signature-based defense techniques such as Firewall are less effective when it comes to proactively defending against Malware primarily because they rely on predefined signature sets and are unprepared against newer malwares. So it is still reactive as opposed to proactive. This research makes use of Honeypots to proactively defend against malware. Jianguo Ding~\cite{behaviorbasedproactive} discussed about analyzing malware behavior to proactively defend against unknown malicious codes. \\
	
	\subsection{Proactive Defense Model based on Cyber Threat Analysis}
	
	Cyber Attacks have become very sophisticated in recent years. Targeted Areas and number of incidents have both increased in alarming quantities. As a result, companies have been forced to adapt an approach where they have to assume that attacks will happen. Sandboxing is one such technique which is used which involves examining files in an isolated environment to check for suspicious behavior. Companies have set up Security Response teams but even they are not enough since damage can happen from leakage of information. The reason for this is APT (Advanced Persistent Threat) which is a repeated targeted attack based on information on security measures. As a result, researchers have begun analyzing APT attacks to develop countermeasures. 
	
	The attack via APT can be described in phases:
	
	\begin{itemize}
		\item First phase is to prepare for attack by using emails with malware to breach security of the targeted organization.
		\item Second phase is to infiltrate PCs with malware.
		\item Third phase is to construct infrastructure by using backdoor and monitoring networks to collect information on the security system and send this info to intruder’s server.
		\item Next phase is to breach and explore servers to get additional information.
		\item Final phase is to steal valuable information.
	\end{itemize}
	An APT Attack is made with a clear objective. According to the researchers, analyzing these attacks can reveal certain patterns about the attackers. These include features of attack infrastructure, tools (malware), techniques and objective. Such information is called “cyber threat intelligence”. The techniques described in this paper are cyber threat analysis techniques and associated cyber threat intelligence. Some of the techniques used are as follows:
	\begin{itemize}
		\item \textbf{Attack Channel Analysis:} Attack Channel is the technique used by an attacker to infect a terminal inside the targeted organization with malware to enable remote operations. This is usually done by sending targeted emails to a terminal in the organization. This email is analyzed to find information such as email source address, Message Subject, Hostname among other things.
		\item \textbf{Malware Analysis:}This has a three step procedure as follows:
		\begin{enumerate}
			\item Surface Analysis - Analyze the file type, file name, characters in file, etc.
			\item Dynamic Analysis - Running the malware and analyzing its behavior to determine file operation, processes, etc.
			\item Static Analysis - This involves reverse engineering the malware.
		\end{enumerate}
		\item \textbf{C2 Analysis:}  This technique analyzes the C2 server to derive information such as information about IP Address, Location, and Domain Names.
	\end{itemize}
	These techniques have led to the standardization of Cyber Threat Intelligence by releasing specifications such as CybOX, STIX and TAXII. Fujitsu’s have come up with a Proactive Defense Model that uses Cyber Threat Intelligence. There are 6 constituent elements of the model and they are as follows:
	\begin{itemize}
		\item SIEM (Security Information and Event Management) - Consolidates events from a variety of sensors and logs from diverse information systems and assesses whether incident has occurred on the basis of a specific pattern or cyber intelligence.
		\item Incident Management - Manages priority and state of incidents and tasks that need to be executed as tickets.
		\item Artifact storage - Saves artifacts of incidents for use in cyber threat analysis.
		\item Automated Engine - Performs analysis, response and storage of cyber threat intelligence.
		\item Cyber threat intelligence Storage environment - Supports CybOX, STIX and TAXII standards.
		\item Cyber threat Analysis Environment - Analyzes threats to extract new cyber threat intelligence for proactive defense.
	\end{itemize}
	As a future work, the researchers plan to make this model more accurate by trying to extract more cyber threat intelligence. Just using this model is insufficient and collaborating with global vendors and communities is essential.
	
	\subsection{HookScout: Proactive Binary-Centric Hook Detection}
	
	Malware registers its own function (hook) in the target location. Later, the data is registered into EIP and execution is redirected to.malware’s own function. The existing hook detection tools have defeated the older techniques but there are several newer ones evolving. What makes function pointer hooking advantageous for the hackers are the following reasons:
	\begin{itemize}
		\item Attack Space is vast with almost 20000 pointers in Windows Kernel.
		\item It is hard to locate and validate with approximately 7000 in dynamically allocated memory regions. Many of them are in polymorphic data structures which are found in Windows Kernel.
	\end{itemize}
	The goal of this technique is as follows: 
	\begin{enumerate}
		\item To automatically generate a hook detection policy taking advantage of the binary distribution of the OS Kernel.
		\item Locate Function Pointers - To deal with polymorphic data structures.
		\item Validate Function Pointers - Only 3\% ever change in their lifetime. So, a simple policy would be to check if constant pointers ever change.
	\end{enumerate}
	HookScout System has the following parts:
	\begin{itemize}
		\item Monitor Engine - The goal is to determine the concrete memory layout. It also determines primitive types for each memory word for each static/dynamic object. The monitor engine also tracks function pointers. It also does the following work:
		\begin{enumerate}
			\item Running the guest OS within TEMU (Tera-emulator) which is a whole-system binary analysis platform within QEMU (Quick emulator).
			\item For dynamic objects, it performs hook memory allocation/deallocation routines.
			\item For static objects, it executes the hook module loading routine.
		\end{enumerate}
		\item Inference Engine - It infers the abstract memory layout for which it uses context-sensitive abstraction. Object creation context is the execution context where an object is created (e.g. Malloc creator). Objects created under the same context have the same type. The solution is to merge the concrete layouts with the same context into an abstract layout.
		\item Detection Engine - Detection Engine enforces the hook detection policy on user’s machine. 
	\end{itemize}
	To evaluate the performance of HookScout, various aspects were evaluated such as Attack Space, Analysis Subsystem and Detection Subsystem. No false alarms were raised during the testing period. Following limitations were uncovered after the experimental evaluation:
	\begin{itemize}
		\item Coverage - 5 \% of the kernel is not covered which could be exploited by the attackers.
		\item Detection Interval - Deciding the detection interval is tough. 5 seconds or even a second might be too late when it comes to detecting attacks.
		\item Uncommon Proprietary Device Drivers - HookScout utilizes QEMU (Quick Emulator) and since other proprietary drivers are never installed, they are not analyzed.
		\item There are limited test cases for the Dynamic Analysis.
		\item Kernel Module can be subverted or misled. Hypervisor is preferred for this reason.
	\end{itemize}
	Function pointer hooking is a new trend and hard to detect but the researchers have developed HookScout which is proactive, binary-centric and context-sensitive.
	
	\subsection{Use of Soft Computing Techniques in Malware Detection}	
	
	Soft computing techniques are widely used in malware detection these days. These techniques have the ability of learning from past incidents and categorize normal and abnormal behavior. Malware includes viruses, worms, trojan horses, spyware and adware. In recent times, network attacks are easy to launch as the tools to perform such attacks are present freely on the internet. To overcome this problem, IDS (Intrusion Detection System) is used against malicious activity. An IDS monitors the system and decides if the activity is sign of an attack or normal activity. Various soft computing and machine learning techniques are employed in IDS to detect various attacks. 
	
	Some of the techniques are as follows:
	\begin{itemize}
		\item Artificial Immune System (AIS) - AIS helps solve complex computational problems. Using learning, feature extraction, and pattern recognition, it offers rich metaphors for its artificial counterpart.
		\item Fuzzy Set - Fuzzy Logic is a means of modelling the uncertainty of the natural language. The use of fuzzy logic is in two areas - algorithms with learning and adaptive capabilities with the purpose of automatically designing fuzzy rules and to enhance readability of machine learning algorithms such as Support Vector Machines or Hidden Markov Model.
		\item Artificial Neural Networks (ANN) - ANN learns to predict the behavior of various users and daemons in the system. The main advantage of ANN is the tolerance to incomplete information and ability to infer solutions from the data without prior knowledge of regularities in the data.
		\item Decision Tree - It is a powerful tool for classification and prediction. It has three main components - nodes, leaves and arcs.
		\item Support Vector Machines - They are a relatively new supervised machine learning technique. Although the basic technique was conceived for binary classification, several methods for single and multi-class problems have been developed. The usefulness of SVM has been already demonstrated in several fields: like pattern recognition, where it can provide optimal statistical classification by means of properly chosen decision functions.
		\item Other techniques - These include genetic algorithms, Evolutionary Algorithms, Swarm Intelligence.
	\end{itemize}
	Malware Detection using Soft Computing Techniques is primarily classified in three categories:
	\begin{itemize}
		\item Misuse/Signature Detection - Researchers have developed many techniques based on detecting signature of a malware. Some are:
		\begin{enumerate}
			\item Signature Verification with SVM - Buffer Overflow Vulnerabilities are exploited by worms which can be handled with network-based Length-based Signature Generator (LESG).
			\item Signature Tree Generation - Network-based Generation has been proposed as a way to automatically and quickly generate signatures for worms, especially polymorphic worms. PolyTree, an NSG system is composed of two sections - signature tree generator and signature selector.
			\item Logical Expression Testing Criteria - Decision Support has problems in accurately identifying attacks. Recognizing that signatures in essence provide the specification of an IDS engine, studying the accuracy of an IDS engine becomes a black-box testing problem.
			\item F-Sign - F-Sign is designed for extracting unique signatures from malware files automatically. Malicious executable is analysed using two approaches: disassembly, utilizing IDA-Pro, and the application of a dedicated state machine in order to obtain the set of functions comprising the executable.
			\item Bayesian Network - It is used for intrusion detection. The flexible nature Bayesian network allows it to be used both for misuse-based and anomaly-based detection process.
			\item Semantic aware Signature Generation - String extraction and matching techniques have been widely used in generating signatures for worm detection, but how to generate effective worm signatures in an adversarial environment still remains a challenging problem. Semantics Aware Statistical (SAS) algorithm have been proposed for automatic signature generation.
		\end{enumerate}
		\item Anomaly Detection - Anomaly Intrusion Detection (IDs) strategy considers abnormal behaviour is rare and tries to model normal rather than anomalous behaviour. Researchers do a lot of work in anomaly detection. Specifically, they use machine-learning algorithm to classify fixed-length patterns generated via sliding window technique to infer the classification of variable length patterns from the aggregation of the machine learning based classification results. Some other techniques proposed by researchers are Feature-Aided Tracking with Hidden Markov Models. Histogram-Based Traffic Anomaly Detection. Anomaly Detection through a Bayesian Support Vector Machine.
		\item Hybrid Detection - Signature based detection system can only detect attacks which are known to system and signatures are defined. Anomaly based detection has higher False-Positive rate. These techniques can be combined to give better results. This combined technique is known as Hybrid detection scheme. A hybrid model is used not only to correlate alerts as accurately and efficiently as possible but also to be able to boost the model in the course of time.
	\end{itemize}
The network data is very large, heterogeneous, highly varying and imbalanced. Real time anomaly detection approaches with high accuracy and low false positive rate are required.
	
	\subsection{A Study of Network Intrusion Detection Based on Proactive Mechanism}
	
	In this paper, the researchers presented the tools and techniques used to protect the networks and its associated resources and requirements for the proactive network intrusion detection mechanism which should be placed in line with the current intrusion detection system to protect the entire network. Though the general impression is the growing cyber security awareness among the masses, but the advanced hacker techniques and sophistication seems to counter the defensive mechanisms easily and fool the users. The malwares propagating in network have become the biggest threat to the increasing internet. \\ \\
	Countermeasures like firewalls or anti-anything (antivirus, anti-spam, anti-spyware, etc.) are all reactive security tools. They are necessary countermeasures and a part of a comprehensive security system, but you must also take action, be proactive, to ensure the highest level of network security. There is a need to put the place a security mechanism to analyse malicious activity without having rely on the traditional signature based tools. To strengthen these signatures based tools and to manage these tools, there is a need to react proactively so that the analysis of the malicious codes can be performed and signatures for the security can be performed. This research will explain the use of proactive security defence mechanism based on different tools and techniques which help in the creation of a tested that would help in testing and identifying the weakness of a network. There are various types of malwares such as Virus, Worms, Trojans, Spyware, Adware and Rootkits. Most recent operating systems come with built in and “enabled by default” firewall package. Starting with Windows XP service pack 2 and since, firewall has been enabled by default on all Microsoft operating systems. Antivirus software is the basic security tool installed in end user computer. They mostly rely on signature based detection where executable files are matched against a signature database of known viruses. New versions have run-time scanning feature that scans the file in real time and avoids execution, if a threat is detected. Signature based detection however results in the antivirus engine failing to detect variants of known viruses, therefore a constant update of antivirus signature database is essential to provide basic protection.\\ \\
	One promising approach to improve network defense is the use of honey pots, a closely monitored computing resource that we want to have probed, attacked or compromised. More precisely, a honeypot is "an information system resource whose value lies in monitoring unauthorized or illicit use of that resource" Honey pots can run any operating system and any number of services. The configured services determine the vectors available to an adversary for compromising or probing the system. Honeypots can be broadly classified into two types - Server and Client Honeypots. Further both of them are classified each into Low interaction and High Interaction Honeypots. Low Interaction Honeypots emulate a variety of host services. In a High Interaction Honeypot, the attacker is given the freedom to interact with a real operating system and their every attempt is logged and accounted for. The second Honeypot category is also classified based on the way they are deployed in a network as follows:
	\begin{itemize}
		\item Production Honeypots - They are placed within an organization’s production network for the purpose of detection. They extend the capabilities of intrusion detection systems.
		\item Research Honeypots - These are deployed by network security researchers – the white hat hackers. They allow complete freedom for the attacker and, in the process; it is possible to learn their tactics.
	\end{itemize}
	The researchers then talk about the way the honeypots are deployed with the design and process flow and conclude by proposing the system of honeypots which is able to detect the malware programs with the help of honeypot as well as by applying the intelligent forensic investigation of the collected network PCAP (Packet Capture) data.
	
	\subsection{Behavior-based Proactive Detection of Unknown Malicious Codes}
	
	Malicious code is defined as any program that is specifically coded to cause an unexpected (and usually) unwanted event on a User’s PC or Server. There are several types of malware such as Adware, Spyware, Virus, Worms, etc. The antivirus technologies that are present currently are Signature-based scanning,	Heuristic analysis, Cyclic redundancy check (CRC) scanner, Vaccination technology, Behavior blockers, Immunizers, Snapshot technology and Sandbox technology.\\ \\
	There has been an increase in the amount of malicious codes along with polymorphism and metamorphism in malicious codes. Researchers have to deal with any reported unknown malicious code by manual analysis, and then denote a signature for it and update the virus database. This paper tries to establish an automatic mechanism to assistant classifying and identifying unknown malicious codes. There are two types of behavioral analysis techniques. One is static analysis in which the binary code is disassembled and the functions of the program are analyzed without executing it. The second one is dynamic analysis in which we analyze the code during runtime. Malicious behavior can be categorized as File-related behavior, Process-related behavior, Window-related behavior, Network-related behavior, Register-related behavior and Windows service behavior.\\ \\
	The paper then describes the analysis methods particularly, Statistic-based method and MoE (Mixture of Experts) Neural Network Model. Two experiments were conducted where the MoE model showed a better performance although both did not get the detection rate to 100\%. Ultimately, the researchers concluded that a single method cannot get the best result.  The detection rate did not reach 100\% because definition of malicious codes changes with users. As a future research, a larger data set was proposed for the experiment and finding other ways to improve proactive detection.	
	
	\section{Tasks Completed}
	This section talks about specific tasks completed by each group member till date.
		\subsection{Tejas Khairnar (1207690220, Group Leader)}
		I, as the group leader, am organizing weekly meetings with the whole team. My duties include keeping a check on work done by every group member.
		I am also helping everyone with finding good papers in the area of 'Malware Analysis Techniques'. Apart from that I have also been reading the research paper "Toward automated dynamic malware analysis using cwsandbox"~\cite{willems2007toward}. \\
		This paper talks about sandbox technique being used for dynamic malware analysis. In dynamic malware analysis, the authors try to execute the malware in a sandboxed environment. They further discuss various approaches they took in order to analyze the malware like API hooking and DLL code injection. They have also built a CWSandboxed architecture in which they have described various phases to achieve malware analysis. The researchers further led their research into an open source tool called as "Cuckoo Sandbox: Automate Malware Analysis". Currently this is the best open source tool available in the security community.\\
		My next steps will be to read more about Malware Analysis techniques and how we can proactively defend and prevent them. In order to build up a concrete story and take the team to successful completion of the research survey I will go through a research survey paper on Automated Dynamic Malware-Analysis Techniques and Tools~\cite{egele2012survey} published in ACM Computing Surveys in the year 2012. 		
		\subsection{Sujay Vaishampayan (1209248166, Deputy Group Leader)}
		I, as the deputy group leader, help the group leader in organizing meetings and keeping track of the progress made by our group. I'm making sure that the deadlines are met and the group is not behind schedule at any point. Currently, I have thoroughly read the paper, "Scalability, Fidelity and Stealth in the DRAKVUF Dynamic Malware Analysis System"~\cite{lengyel2014scalability}. \\
		Implementing an intrusion detection system on computers may tend to be in effective at times as malware has access to the entire system and may render the IDS ineffective. The malware will stealthily execute it's tasks and delete itself from the system or copy itself on to another system without leaving a single trace on the affected computer. This paper proposes a novel way in which we make use of hardware virtualization extension such as the Xen virtual Machine Monitor (VMM). We run multiple virtual machines on this hypervisor and the memory of each virtual machine is analyzed in the 'dom0' of the Xen hypervisor. DRAKVUF makes use of Copy-on-Write method to examine the memory by efficiently transferring the part of the memory which it wishes to examine over the network. The virtual machines are all connected along the same network via a NAT engine which makes the virtual machines invisible to each other. This effectively reduces the number of attack points or intrusion points for the malware and each virtual machine functions as an independent node. DRAKVUF makes use of a Direct Memory Access technique using LibVMI to access the memory of the target machine.\\ 
		Implementing a dynamic malware analysis system using hardware virtualization extensions along with virtual machine introspection is a novel approach. This enables almost complete isolation of the malware and the IDS which greatly increases the probability of identifying threats.\\
		I will be studying and analyzing the following papers in detail in the future:
		\begin{itemize}[noitemsep]
			\item Ether: Malware Analysis via Hardware Virtualization Extensions~\cite{dinaburg2008ether}
			\item Process Implanting: A New Active Introspection Framework for Virtualization~\cite{jiang2011procimplant}
			\item SPIDER: Stealthy Binary Program Instrumentation and Debugging via Hardware Virtualization~\cite{dongyan2013spider}
		\end{itemize}		
		
		\subsection{Zhibo Sun(1207644187)}
		In order to have a new point of view to proactively predict and prevent malwares, we are not only considering the analysis of the malware, but also the threat and malware intelligence. In order to have knowledge in this section, the paper I read was "Needles in a Haystack: Mining Information from Public Dynamic Analysis Sandboxes for Malware Intelligence~\cite{graziano2015needles}" that was published in 2015 USENIX SECURITY.
		
		This paper proposed a novel methodology to automatically detect if miscreants submit their samples to malware analysis sandbox during the malware developments phase and if this is the case, to acquire more insights about the dynamics of malware development. Their experimental results show that: by combining dynamic and static analysis with features based on the file submission, it is possible to achieve a good accuracy in automatically identifying cases of malware developments. They are able to automatically identify thousands of developments and show how the authors modify their programs to test their functionalities or to evade detections from known sandboxes. The more important contribution is that they provide a new point to proactively predict and prevent malicious softwares through threat intelligence, instead of only focusing on malware analysis.\\
		My next steps will be to explore more about threat intelligence and how malware analysis can be a part of it. I will summarize in-depth about our paper "Toward Automated Threat Intelligence Fusion"~\cite{moditowards} and explain strategies for proactive detection and prevention against Malware threats.
		
		\subsection{Harshil Maskai (1209174092)}
		My team is working collectively towards understanding the various aspects of Malware Analysis. I have been assigned the task of researching the topic of Malware Reverse Engineering and gathering as much literature on this topic as possible. I started by reading the paper "Helping Johnny to Analyze Malware - A usability-optimized decompiler and malware analysis user study~\cite{yakdan2016helping}" which talks about the current industry standards being used to decompile malware code and their shortcomings. \\
		The two major contribution made by this paper were:
		First, the authors provided several code transformation techniques which preserve the semantics of the malware code and improve code readability. The authors created DREAM++ an extension to the current state-of-the-art decompilers which simplifies expression and control flow transformations and also maintains the semantics of the variables and constants according to the context they were used in.
		Second, the authors conducted several studies to compare the three decompilers and note the improvements of the DREAM++ over the already existing decompilers.\\
		My next steps involve diving deep into the topic of Malware reverse engineering by reading papers associated to the one above. 
		The associated papers which I found useful and relevant to this one are:
		\begin{itemize}[noitemsep]
			\item{Decompilers and beyond~\cite{guilfanov2008decompilers}}
			\item{No More Gotos: Decompilation Using Pattern-Independent Control-Flow Structuring and Semantics-Preserving Transformations~\cite{yakdan2015no}}
			\item{Automatic Reverse Engineering of Data Structures from Binary Execution~\cite{lin2010automatic}}
		\end{itemize}
		
		\subsection{Aloma Lopes (1209273126)}
		I have been researching on processes to determine the behavior and purpose of any malware sample and thus went through the paper “Exploring Multiple Execution Paths for Malware Analysis~\cite{moser2007exploring}”. Traditional Malware analysis tools observe the behavior of the sample only on a single program execution path, i.e, the reports generated by their analysis of the samples contains the interaction observed in a given test environment and at a particular point in time. This does not give us a comprehensive overview of the actions that the sample can perform. The reason is that these malware programs contain certain checks for the execution of certain parts of their code and if these conditions are not met while it’s interaction in that execution trace in which it is analyzed, incorrect conclusions may be drawn as these conditional parts of the code have not been executed.\\
		According to the paper, the limitation of these traditional malware analysis tools can be overcome by increasing the test coverage and allow automated malware systems to explore multiple execution paths of a malware program, depending on how certain inputs are processed by the code. It tracks some interesting inputs that are read by the program and then identify decision points that use these inputs to decide the program flow. These decision points are then snapshot, so that we return to them to alter the inputs in such a way that it explores all possible branches. The experimental results performed on a set of 308 real-world malicious code samples demonstrate that these different malware samples exhibit different behavior based on the input read from the environment.\\
		To further understand existing systems that perform malware analysis, I will read the following papers and understand the procedure that they have followed to detect the behavior of malware samples:
		\begin{itemize}[noitemsep]
			\item TTAnalyze: A Tool for Analyzing Malware~\cite{bayer2006ttanalyze}
			\item Towards Automatically Identifying Trigger-based Behavior in Malware using Symbolic Execution and Binary Analysis~\cite{brumley2008automatically}
			\item Detecting Kernel-Level Rootkits Through Binary Analysis~\cite{kruegel2004detecting}
		\end{itemize}
	
		\subsection{Chaitanya Palaka (1209261868)}
		To understand the topic of Malware Analysis better, I had chosen to read the paper 'Linux kernel-based feature selection for Android Malware Detection~\cite{kim2014linux}'. This paper discusses the automatic detection of malware on Android systems. Specifically, malware detection on Android was chosen because the platform is more vulnerable than others due to its open source nature. The authors of this paper use Machine Learning to achieve this, using a custom feature selection method and an SVM (support vector machine) classifier. The authors have composed a set of 59 features from the linux kernel from the subcategories of Memory, CPU and Network. These 'features' were taken from the linux /proc folder which holds system process information. The authors have then chosen to test their classifier with different set of features to test the accuracy. This was also due to the fact that previous research in this area used a far less number of features, and the authors were mainly considering the question : at what number of features does the classifier perform optimally? The results of their experiments were quite good, and optimal and correct classification of malware was found with 36 unique features.\\
		Going forward, I will research more about malware detection on linux systems. The papers i am considering reading and understanding are:
		\begin{itemize}[noitemsep]
			\item Mobile malware detection through analysis of deviations in application network behavior.~\cite{shabtai2014mobile}
			\item Andromaly”: a behavioral malware detection framework for android devices.~\cite{shabtai2012andromaly}
			\item AppsPlayground: automatic security analysis of smartphone applications.~\cite{rastogi2013appsplayground}
			\item DroidScope: Seamlessly Reconstructing the OS and Dalvik Semantic Views for Dynamic Android Malware Analysis.~\cite{yan2012droidscope}
		\end{itemize}
		
		\subsection{Varun Chandrasekar (1209248010)}
		Our team is working towards understanding various techniques that aid in malware analysis and detection. With respect to this have taken up the task of studying the research paper titled "Fileprint Analysis for Malware Detection~\cite{stolfo2005fileprint}". The paper elucidates the various experiments the authors conducted to augment existing principles in malware detection and specifically talks about employing statistical binary content analysis to better single out the file segments that appear to be infected by malware. The experiments conducted included inserting malware at different locations and seeing how easily it is detected by the scanner with respect to the position it was injected in. It also talks about analyzing binary content 1-gram distributions to classify infected files and also talks about using n-gram distance to distinguish self encrypted files from common executable files.\\
		Overall the paper gives a good understanding of the various pros and cons to the current malware detection techniques and discusses the results of experiments that were conducted to augment them. To further deepen my understanding, and collect as much relevant information on the topic as possible, I plan to go through the research papers listed below:
		\begin{itemize}[noitemsep]
			\item Fileprints: Identifying File Types by n-gram Analysis.~\cite{li2005fileprints}
			\item MEF: Malicious Email Filter A UNIX Mail Filter that Detects Malicious Windows Executables~\cite{schultz2001mef}
			\item Content Based File Type Detection Algorithms~\cite{mcdaniel2003content}
		\end{itemize}
		\subsection{Vimal Vadivelu (1209203043)}
		My work is to find good research papers in the malware analysis field and understand them in detail. In addition to the above, I also read a paper, “Identifying Dormant Functionality in Malware Programs~\cite{comparetti2010identifying}”, which proposes a new method in dynamic malware analysis. 
		This paper proposed an approach that leverage behavior observed while dynamically executing a specific malware sample to identify similar functionality in other programs. When they observe malicious actions during dynamic analysis, they automatically extract and model the parts of the malware binary that are responsible for this behavior. Then, they leverage these models to check whether similar code is present in other samples, which allows them to identify dormant functionality( functionality that is not observed during dynamic analysis) statically in malicious program.\\
		The paper has three main contributions:
		Introduce a novel technique to automatically identify and model code regions in binaries that are directly responsible for specific runtime behaviors,
		Present a system that leverages models to statically check unknown programs for the presence of previously-seen, malicious functionality,
		Experimental evaluation demonstrates that the system successfully finds dormant behaviors in malware samples that are not discovered by a dynamic malware analysis tool.
		
		
		Overall the paper gives good understanding of dynamic malware analysis and its advantages and disadvantages. To further deepen my knowledge, the papers I am further considering are,
		\begin{itemize}[noitemsep]
			\item Integrating Static and Dynamic Malware Analysis Using Machine Learning~\cite{mangialardo2015integrating}
			\item PyTrigger: A System to Trigger \& Extract User-Activated Malware Behavior~\cite{fleck2013pytrigger}
			\item A Testing Model for Dynamic Malware Analysis Systems~\cite{massicotte2012testing}
		\end{itemize}
		
		\subsection{Kunal Bansal (1211213169)}
		My work is based on on providing malware network analysis. I went through “Sandnet: Network Traffic Analysis of Malicious Software~\cite{rossow2011sandnet} paper, in which researchers are providing analysis about using network level activities to find unique vectors that can be used to backup insights provided by system level activities. These unique behaviors can be further used to collect and classify information and eventually mitigate malicious software.\\
		They have created an in-depth analysis of malware network behaviour using the Sandnet system over a period of 12 months. Sandnet counters two limitations, short analysis period and lack of detailed network behavior analysis which provides an overview of network protocols used by the malware, namely DNS and HTTP. The paper achieves the following things, a Sandnet system for data collection using network analysis which can be used for prolonged period of time, a network activities’ overview of 100,000 malwares and augment the data with previous efforts.
		and an in-depth analysis of DNS and HTTP traffic to define protocol specific usage behaviours of malware.\\
		This paper also provides an analysis of basic vulnerability detection in a system using network analysis. This is a dynamic and data analytics based approach to use data mining using software and analyze malwares based on that. I would be exploring more software based network analysis tools to detect malware in cloud, I would be going through the following research papers hereafter:
		\begin{itemize}[noitemsep]
			\item Snort: Lightweight intrusion detection for networks.~\cite{roesch1999snort}
			\item NICE: Network intrusion detection and countermeasure selection in virtual network systems.~\cite{chung2013nice}
		\end{itemize}

		\subsection{Raj Dalvi (1209232176)}
		My responsibility is to work with my team in ensuring the timely submissions of all deliverables. I also went through the paper “Siren: Catching Evasive Malware~\cite{borders2006siren}” and got more information on proactive Malware detection and techniques for preventing damage via new malware.\\
		The paper elaborates on a tool called Siren, an activity injection system which intends to fool malicious attackers by mocking human actions so the attacker will focus on them instead of the actual human actions. The traditional behavioral analysis is able to detect a lot of novel threats but they are always a step behind because they are reactive rather than proactive. Siren attempts to be more proactive against malware, specifically spyware.\\
		Secondly, the paper also admits that the tool has some weaknesses such as using an out-of-band channel to detect the mimicry of human activities. So, it is a well-rounded paper and provides a good perspective on proactive malware analysis.
		
		The researchers used virtual machines for their experiments. So, I will go into further detail on how they performed said experiments by going through the following papers:
		\begin{itemize}[noitemsep]
			\item Subvirt: Implementing malware with virtual machines.~\cite{king2006subvirt}
			\item An Evening with Berferd In Which a Hacker is Lured Endured and Studied.~\cite{cheswick1992evening}
			\item Undermining an Anomaly-Based Intrusion Detection System Using Common Exploits.~\cite{tan2002undermining}
		\end{itemize}
	

	\section{Results}
	Till now we have distributed various areas of Malware Analysis Techniques and proactive detection \& prevention amongst the group members. This distribution was done after every group member summarized their first research paper read about Malware Analysis. Every group member is taking equal efforts in understanding the topic they are working on and summarizing their work.\\ 
	Table 1 shows various focus areas each group member is working on. Table 2 shows our project timeline we are trying to follow in order to complete our work in timely manner.
	\begin{center}
		\begin{tabular}{ccc}
			\hline
			Sr. No & Name & Research Area\\
			\hline
			1 & Tejas Khairnar & Proactive Detection and Prevention of Malwares\\
			\hline
			2 & Sujay Vaishampayan & Malware Analysis using Hardware Virtualization	Techniques\\
			\hline
			3 & Zhibo Sun & Proactive Detection and Prevention of Malwares\\
			\hline
			4 & Harshil Maskai & Malware Reverse Engineering\\
			\hline
			5 & Aloma Lopes & Dynamic Malware Analysis\\
			\hline
			6 & Chaitanya Palaka & Malware Analysis using Machine Learning\\
			\hline
			7 & Varun Chandrasekar & Malware Reverse Engineering\\
			\hline
			8 & Raj Dalvi & Proactive Malware Detection\\
			\hline
			9 & Vimal Vadivelu & Dynamic Malware Analysis\\
			\hline
			10 & Kunal Bansal & Dynamic Malware Analysis\\
			\hline
	\end{tabular}\\
%	\caption{Table 1: Research Areas} \label{table: Research Areas}
	\end{center}
	\begin{center}
		\begin{tabular}{cccc}
			\hline
			Sr. No & Name & Timeline & Status\\
			\hline
			1 & Decide about Research topic  & Week 1 starting 9th January & Completed\\
			\hline
			2 & Search for research papers related to the topic & Week 2 starting 16th January & Completed\\
			\hline
			3 & Intial Project Proposal & Week 3 starting 23rd January & Completed\\
			\hline
			4 & Summarize the work done till date for interim report & Week 3 \& 4 till 10th February & Completed\\
			\hline
			5 & Interim Project Report  & Week 6 starting 20th February  & Completed\\
			\hline
			6 & Reading more literature about Malware analysis techniques & Week 7 starting 27th February & TBD\\
			\hline
			7 & Discuss more about proactive detection and prevention of Malwares & Week 9 starting 13th March & TBD\\
			\hline
			8 & Compiling more data from research papers read & Week 10 starting 20th March & TBD\\
			\hline
			9 & Compiling the final survey report & Week 11 till 27th March & TBD\\
			\hline
		\end{tabular}\\
%	\caption{Table 2: Project Timeline} \label{table: Project Timeline}
	\end{center}
\bibliographystyle{plain}
\bibliography{biblio}
\end{document}

