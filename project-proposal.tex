\documentclass[14pt]{article}
\usepackage{graphicx}
\usepackage{epsfig}
\usepackage{url}
\usepackage[english]{babel}
\usepackage{vmargin}
\usepackage{times}
\usepackage{amssymb}
\usepackage[fleqn]{amsmath}
\usepackage{cite}
\usepackage{titling}
\usepackage{color}
\usepackage{xspace}
\usepackage{listings}
\usepackage{upquote}
\usepackage[hidelinks]{hyperref}
\usepackage{wrapfig}
\setpapersize{USletter}
\usepackage{textpos}

\setlength{\textheight}{220mm}
\setlength{\textwidth}{160mm}
\evensidemargin=1.1in
\oddsidemargin=1.1in
\topmargin=0.45in
\usepackage[compact,noindentafter]{titlesec}
\titlespacing{\section}{0pt}{*2.4}{*1.8}
\titlespacing{\subsection}{0pt}{*2.0}{*1.6}
\titlespacing{\subsubsection}{0pt}{*1.6}{*1.2}

\newcommand{\fulltitle}{Malware Analysis for Proactive Detection and Prevention\xspace}
\title{\fulltitle}
\author{Tejas Khairnar (1207690220, Group Leader) \\Sujay Vaishampayan (1209248166, Deputy Group Leader)\\  Zhibo Sun(1207644187)\\Harshil Maskai (1209174092)\\ Aloma Lopes (1209273126)\\ Chaitanya Palaka (1209261868)\\ Varun Chandrasekar (1209248010) \\ Kunal Bansal (1211213169) \\ Raj Dalvi (1209232176) \\ Vimal Vadivelu (1209203043)\\ \\
		Arizona State University}

\begin{document}
	\begingroup
		\fontsize{15pt}{15pt}\selectfont
		\begin{center}
			CSE 543 Information Assurance and Security \\~\\
			Interim Report
		\end{center}
	\endgroup
	{\let\newpage\relax\maketitle}
	\section{Introduction}
	\subsection{Background and Motivation}
	In today’s world connected by the Internet, cyber security has become a major concern. In order
to be secure, software as well as hardware industries around the world are working proactively to
secure their software and stay secured. One of the major threats to these industries arises due to
malware which can be defined as a small piece of software that is intended to damage or disable
computers and computer systems. Hence our research survey is focused on studying techniques
to proactively detect and defend against these malwares.

Stuxnet[cite] considered to be the most sophisticated piece of Malware ever detected was used in operations against Iran in 2010. This Malware was so sophisticated and complicated that, it was believed that this can be developed and deployed only with fundings which can be compared nationwide. Stuxnet is unique in its own nature because unlike other Viruses or Worms it targets systems that are traditionally not connected to the internal network, which can also be termed as isolated machines. It infected Windows machines through USB keys and then propagated accross the network scanning for Siemens Step7 software on computers controlling a PLC (programmable logic controllers). This helped the Malware to use the information it gathered to take control of other systems, and making them behave to failure. In order to maintain its persistent it provided false feedback to the main controllers outside the network. All these efforts were taken to slow down Iranian Nuclear Program which was considered as a cyber war.

The above example inspires us about the need to be aware about cyber security and how it might affect our cyber world. Many organizations like Kaspersky labs, CorwdStrike are constantly developing expertise to detect these kind of Malwares and Trojans spreading around the world. Therefore, we find this topic Malware analysis for proactive detection and prevention very intriguing to study about.
	\subsection{Goals and scope}
	Our goal for this research survey is to explore and scrutinize various techniques to perform Malware Analysis and understand its application in the real world for proactive defense.
	The scope of our research survey encompasses the following areas:
	\begin{itemize}
		\item{Various methods to perform Malware Analysis}
		\item{Tools currently used to perform Malware Analysis} 
		\item{Techniques for proactive defense against Malware}
		\item{Techniques to prevent IT networks from Malwares}
	\end{itemize}
	
	\section{Tasks Completed}
	This section talks about specific tasks completed by each group member till date.
		\subsection{Tejas Khairnar (1207690220, Group Leader)}
		\subsection{Sujay Vaishampayan (1209248166, Deputy Group Leader)}
		\subsection{Zhibo Sun(1207644187)}
		\subsection{Harshil Maskai (1209174092)}
		\subsection{Aloma Lopes (1209273126)}
		\subsection{Chaitanya Palaka (1209261868)}
		\subsection{Varun Chandrasekar (1209248010)}
		\subsection{Kunal Bansal (1211213169)}
		\subsection{Raj Dalvi (1209232176)}
		\subsection{Vimal Vadivelu (1209203043)}
	
	\section{Results}
	
	\section{Responsibilities}
	\begin{table*}[t]
		\centering
		\begin{tabular}{lrrrrrrr}
			\hline
		\end{tabular}
	\end{table*}
\bibliography{biblio}
\end{document}

